\chapter{Preface}

\raggedright
\setlength{\parskip}{0.5em}

\textbf{Note:} Visit \url{http://github.com/Alextrovert/Algorithm-Anthology} for the most up-to-date digital version of this codebook. The version you are reading is currently being reviewed, revised, and rewritten.

\section{Introduction}

This anthology started as a personal project to implement common algorithms in the most concise and "vanilla" way possible so that they're easily adaptable for use in algorithm competitions. To that end, several properties of the algorithm implementations should be satisfied, not limited to the following:
\begin{itemize}
	\item Implementations must be clear. There is no time to write rigorous documentation within contests. This makes it all the more important to make class and variable names reflexive of what they represent. Clarity must also be carefully balanced with not making them too long-winded, since it can be just as time-consuming to type out long identifiers.
	\item Implementations must be generic. The more code that must be changed during the contest, the more room there is for mistakes. Thus, it should be easy to apply implementations to different purposes. C++ templates are often used to accomplish this at the slight cost of readability.
	\item Implementations must be portable. Different contest environments use different versions of C++ (though almost all of them use GCC), so in order to make programs as compatible as possible, non-standard features should be avoided. This is also why no features from C++0x or above are used, since many constest systems remain stuck on older versions of the language. Refer to the ``Portability" section below for more information.
	\item Implementations must be efficient. The code cannot simply demonstrate an idea, it should also have the correct running time and a reasonably low constant overhead. This is sometimes challenging if concision is to be preserved. However, contest problem setters will often be understanding and set time limits liberally. If an implementation from here does not pass in time, chances are you are choosing the wrong algorithm.
	\item Implementations must be concise. During timed contests, code chunks are often moved around the file. To minimize the amount of scrolling, code design and formatting conventions should ensure as much code fits on the screen as possible (while not excessively sacrificing readability). It's a given that each algorithm should be placed within singleton files. Nearly all contest environments demand submissions to be contained within a single file.
\end{itemize}

A good trade-off between clarity, genericness, portability, efficiency, and concision is what comprises the ultimate goal of adaptability.

\section{Portability}

All programs are tested with version 4.7.3 of the GNU Compiler Collection (GCC) compiled for a 32-bit target system.

That means the following assumptions are made:

\begin{compactitem}
	\item bool and char are 8-bit
	\item int and float are 32-bit
	\item double and long long are 64-bit
	\item long double is 96-bit
\end{compactitem}

Programs are highly portable (ISO C++ 1998 compliant), \textbf{except} in the following regards:

\begin{compactitem}
	\item Usage of long long and related features \texttt{[-Wlong-long]} (such as \texttt{LLONG\_MIN} in \texttt{\textlangle{}climits\textrangle{}}), which are compliant in C99/C++0x or later. 64-bit integers are a must for many programming contest problems, so it is necessary to include these.
	\item Usage of variable sized arrays \texttt{[-Wvla]} (an easy fix using vectors, but I chose to keep it because it is simpler and because dynamic memory is generally good to avoid in contests)
	\item Usage of GCC's built-in functions like \texttt{\_\_builtin\_popcount()} and \texttt{\_\_builtin\_clz()}. These can be extremely convenient, and are easily implemented if they're not available. See here for a reference: \url{https://gcc.gnu.org/onlinedocs/gcc/Other-Builtins.html}
	\item Usage of compound-literals, e.g. \texttt{vec.push\_back((mystruct)\{a, b, c\})}. This is used in the anthology because it makes code much more concise by not having to define a constructor (which is trivial to do).
	\item Ad-hoc cases where bitwise hacks are intentionally used, such as functions for getting the signbit with type-puned pointers. If you are looking for these features, chances are you don't care about portability anyway.
\end{compactitem}


\section{Usage Notes}

The primary purpose of this project is not to better your understanding of algorithms. To take advantage of this anthology, you must have prior understanding of the algorithms in question. In each source code file, you will find brief descriptions and simple examples to clarify how the functions and classes should be used (not so much how they work). This is why if you actually want to learn algorithms, you are better off researching the idea and trying to implement it independently. Directly using the code found here should be considered a last resort during the pressures of an actual contest.

All information from the comments (descriptions, complexities, etc.) come from Wikipedia and other online sources. Some programs here are direct implementations of pseudocode found online, while others are adaptated and translated from informatics books and journals. If references for a program are not listed in its comments, you may assume that I have written them from scratch. You are free to use, modify, and distribute these programs in accordance to the license, but please first examine any corresponding references of each program for more details on usage and authorship.

Cheers and hope you enjoy!

\begin{flushright}
--- Alex Li\\[0.3\baselineskip]
December, 2015
\end{flushright}

